\begin{tikzpicture}
  \tikzset{x=10pt,y=10pt}
  %%%%
  % Robot

  % Robot's vertices definition
  \coordinate (a1) at (5,0);
  \coordinate (a2) at (1,4);
  \coordinate (a3) at (-3.5,-1.0);
  \coordinate (a4) at (0,-3);
  \node[green,right=0.5em] at (a1){$a_1$};
  \node[green,right=0.5em] (ve) at (a2){$a_2$};
  \node[green,left=0.5em] at (a3){$a_3$};
  \node[green,below=0.5em] at (a4){$a_4$};

  % Robot's polygon
  \draw[line join=round,green,fill=green!20,line width=\vRad] (a1) -- node[right]
  {$e_{1,2}$} (a2) -- node[left] {$e_{2,3}$} (a3) -- node[below=0.2em]
  {$e_{3,4}$} (a4) -- node[below=0.2em] {$e_{4,1}$} (a1) -- cycle;
  \node[green] at (barycentric cs:a1=0.25,a2=0.25 ,a3=0.25,a4=0.25){$\mathcal{A}$};

  % Robot's vertices plotting
  \foreach \pts in {a1,a2,a3,a4}
  \fill [green!70] (\pts) circle (2*\vRad);

  %%%%%
  % 1st obstacle

\only<1>{
  \def\oAngle{101}
  % Defining the vertices relative to (a2), and making sure it is tangent.
  \coordinate (a2tmp) at ($(a2)+(\oAngle:2.5*\vRad)+(0,0.51)$);
  \coordinate (b11) at ($(a2tmp)+(\oAngle+90:3)+(0,0.51)$);
  \coordinate (b21) at ($(a2tmp)+(\oAngle-90:2)+(0,0.51)$);
  \coordinate (b31) at ($(a2tmp)+(\oAngle+5:4.5)+(0,0.51)$);
  \node [red,left=0.2em] at (b11) {$b_1^1$};
  \node [red,right=0.2em] at (b21) {$b_2^1$};
  \node [red,above=0.2em] at (b31) {$b_3^1$};

  % 1st obstacle's polygon
  \draw[line join=round,red!90,fill=red!20,line width=\vRad] (b11) -- (b21) -- (b31)  -- cycle;
  \node[red] at (barycentric cs:b11=0.333,b21=0.333,b31=0.333){$\mathcal{O}_1$};

  \foreach \pts in {b11,b21,b31}
  \fill [red!70] (\pts) circle (2*\vRad);

  %%%%%
  % 2nd obstacle
  \coordinate (b12tmp) at ($(a4)!(10,-11)!(a1)+(0,-0.2)$);
  \coordinate (b12) at ($(b12tmp)!2.5*\vRad!(10,-11)$);
  \node[above right] (ev) at (b12){};
  \coordinate (b22) at ($(b12)+(355:5)$);
  \coordinate (b32) at ($(b12)+(290:5.2)$);

  \draw[line join=round,red!90,fill=red!20,line width=\vRad] (b12) -- (b22) -- (b32) -- cycle;
  \node[red] at (barycentric cs:b12=0.333,b22=0.333,b32=0.333){$\mathcal{O}_2$};

  \foreach \pts in {b12,b22,b32}
  \fill [red!70] (\pts) circle (2*\vRad);
}

%% Colision!!!
\only<2>{
  \def\oAngle{101}
  % Defining the vertices relative to (a2), and making sure it is tangent.
  \coordinate (a2tmp) at ($(a2)+(\oAngle:2.5*\vRad)-(0,0.51)$);
  \coordinate (b11) at ($(a2tmp)+(\oAngle+90:3)-(0,0.51)$);
  \coordinate (b21) at ($(a2tmp)+(\oAngle-90:2)-(0,0.51)$);
  \coordinate (b31) at ($(a2tmp)+(\oAngle+5:4.5)-(0,0.51)$);
  \node [red,left=0.2em] at (b11) {$b_1^1$};
  \node [red,right=0.2em] at (b21) {$b_2^1$};
  \node [red,above=0.2em] at (b31) {$b_3^1$};

  % 1st obstacle's polygon
  \draw[line join=round,red!90,fill=red!20,line width=\vRad,opacity=0.7] (b11) -- (b21) -- (b31)  -- cycle;
  \node[red] at (barycentric cs:b11=0.333,b21=0.333,b31=0.333){$\mathcal{O}_1$};

  \foreach \pts in {b11,b21,b31}
  \fill [red!70] (\pts) circle (2*\vRad);

  %%%%%
  % 2nd obstacle
  \coordinate (b12tmp) at ($(a4)!(10,-11)!(a1)-(0,-0.2)$);
  \coordinate (b12) at ($(b12tmp)!2.5*\vRad!(10,-11)$);
  \node[above right] (ev) at (b12){};
  \coordinate (b22) at ($(b12)+(355:5)$);
  \coordinate (b32) at ($(b12)+(290:5.2)$);

  \draw[line join=round,red!90,fill=red!20,line width=\vRad] (b12) -- (b22) -- (b32) -- cycle;
  \node[red] at (barycentric cs:b12=0.333,b22=0.333,b32=0.333){$\mathcal{O}_2$};

  \foreach \pts in {b12,b22,b32}
  \fill [red!70] (\pts) circle (2*\vRad);
}

%%% Contacts
% \only<3>
\end{tikzpicture}